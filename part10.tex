% !TEX root = RailVehicleIntroduction.tex
\section{Vehicle design}

\subsection{Carbody design}
\frame{\frametitle{Requirements on the car body}
\framesubtitle{}
\begin{itemize}
\item  Strength (e.g. EN 12663):
\begin{itemize}
	\item Compressive/tensile force in train
	\item Crashworthiness (e.g. EN 15227)
	\item Pressure shocks, tightness
	\item Forces due to load
	\item Oscillations
\end{itemize}
\item Customer and operational requirements
\begin{itemize}
	\item Life cycle
	\item Maintainability
	\item Lightweight design
	\item Recycling
\end{itemize}
\item Normative Requirements
\begin{itemize}
	\item Fire \& Smoke (e.g. EN 45545)
	\item Toxiticity (e.g. EG 1907/2006 REACH)
\end{itemize}
\end{itemize}
}

\frame{\frametitle{Main elements of the car body}
\framesubtitle{}
\begin{columns}[t] 
     \begin{column}[T]{6cm} 
     	\begin{itemize}
     		\item Underframe
		\begin{itemize}
		\item Central or side rails
		\item Beams
		\end{itemize}
		\item Side walls
		\item Roof
		\item Front wall
		\item Cab module
   	\end{itemize}
     \end{column}
     	\begin{column}[T]{6cm} 
         	\begin{center}
            		\includegraphics[width=0.8\textwidth]{FrontNose}
		\rotatebox{90}{\tiny \color{gray}{Source: Voith Press Image}}
        		\end{center}
     \end{column}
 \end{columns}
}


\frame{\frametitle{Design principles of car body design}
\framesubtitle{}
\begin{columns}[t] 
\begin{column}[T]{.5cm}
\end{column} 
     \begin{column}[T]{5.5cm} 
     \textbf{Differential design} \vspace{.2cm}
     	\begin{itemize}
     		\item Manufacturing from semis:
		\begin{itemize}
		\item Parts of simple shape
		\item Car shape due to joining and bending
		\end{itemize}
     	\end{itemize}
	\textbf{Integral design}
     	\begin{itemize}
     		\item Manufacturing from complex shaped parts:
		\begin{itemize}
		\item e.g. extruded profiles
		\item Car shape due to joining and cutting
		\end{itemize}
     	\end{itemize}
	\textbf{Load bearing}
     	\begin{itemize}
     		\item Supporting frame
		\item Self supporting car body
	\end{itemize}
     \end{column}
     	\begin{column}[T]{6cm} 
         	\begin{center}
            		\includegraphics[width=0.8\textwidth]{Integral}\rotatebox{90}{\tiny \color{gray} Source: Siemens Press Image}
        		\end{center}
     \end{column}
 \end{columns}
}


\frame{\frametitle{Car body materials}
\framesubtitle{}
\begin{itemize}
\item Stahl:
\begin{itemize}
	\item Classic: engineering steels E235, E355
	\item For particular applications: stainless steels, e.g. X5CrNi18-10
	\item Well suited for joining
	\item Fatigue strength and elastic/plastic behaviour well behaved
\end{itemize}
\item Aluminium:
\begin{itemize}
	\item Lower density, lower strength
	\item Fatigue limit more difficult
	\item Welding beads susceptible to fatigue
\end{itemize}
\item Plastics:
\begin{itemize}
	\item Typically fibre reinforced (Glass, Carbon)	
	\item Enable integral design and integration of functions
	\item Also in form of sandwich materials 
\end{itemize}
\item Honeycomb and foams:
\begin{itemize}
	\item Applied mainly in crash deformation area
\end{itemize}
\end{itemize}
}


\frame{\frametitle{Lightweight design of car bodies}
\framesubtitle{}
\begin{itemize}
\item Make all components load bearing
\item Well bearable:
\begin{itemize}
	\item Longitudinal forces
\end{itemize}
\item Needs additional cross section:
\begin{itemize}
	\item Torques and moments
\end{itemize}
\item High strength materials slowly accepted
\begin{itemize}
	\item Doubts on maintainability and life cycle
\end{itemize}
\end{itemize}
\begin{center}
\includegraphics[width=0.8\textwidth]{Class66}
\end{center}
}


\subsection{Loading gauge}

\frame{\frametitle{Loading gauge}
\framesubtitle{}
%\begin{columns}[t] 
%     \begin{column}[T]{6cm} 
     	\begin{itemize}
		\item Load gauge needs to encompass
		\begin{itemize}
		\item Loading condition
		\item Dynamical movements
		\item Curving
		\item Compatibility to other vehicles
		\end{itemize}
     	\end{itemize}
%     \end{column}
%     	\begin{column}[T]{6cm} 
         	\begin{center}
            		\includegraphics[width=0.7\textwidth]{G1G2}\source{Source: Christian Lindecke}
        		\end{center}
%		\end{column}
%		\end{columns}
     }


%\offslide{Breiteneinschr\"ankung und Lichtraumbedarf}

\frame{\frametitle{Wheelset and linear loads}
\framesubtitle{}
\begin{columns}[t] 
     \begin{column}[T]{6cm} 
     	\begin{itemize}
     		\item Limitation of wheelset (axle) load:
		\begin{itemize}
		\item Acc. to track category
		\item Normative, e.g. TSI Loc\&Pas (for HSR), EN 15528
		\end{itemize}
		\item Limitation of linear load
		\begin{itemize}
		\item e.g. bridges, track ballast
		\end{itemize}
     	\end{itemize}
     \end{column}
     	\begin{column}[T]{6cm} 
         	\begin{center}
		\begin{tabular}{|c|c|c|}
		\hline
			Class & Wheelset load & Linear load \\ \hline
			A & 16 t & 5{,}0 t/m \\ \hline
			B1 & 18 t & 5{,}0 t/m \\ \hline
			B2 & 18 t & 6{,}4  t/m \\ \hline
			C2 & 20 t & 6{,}4  t/m \\ \hline
			C3 & 20 t & 7{,}2  t/m \\ \hline
			C4 & 20 t & 8{,}0  t/m \\ \hline
			D2 & 22{,}5 t & 6{,}4  t/m \\ \hline
			D3 & 22{,}5 t & 7{,}2  t/m \\ \hline
			D4 & 22{,}5 t & 8{,}0  t/m \\ \hline
			E4 & 25 t & 8{,}0  t/m \\ \hline
			E5 & 25 t & 8{,}8  t/m \\ \hline
		\end{tabular}
        		\end{center}
     \end{column}
 \end{columns}
}

