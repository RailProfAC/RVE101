% !TEX root = RailVehicleIntroduction.tex
\addtocontents{toc}{\protect\newpage}
\section{Track guidance}
\subsection{Track gauges}

\frame{
\frametitle{Track gauge}
\framesubtitle{}
\begin{columns}[t] 
     \begin{column}[T]{5cm} 
     	\begin{itemize}
     		\item Track gauge
		\begin{itemize}
		\item Based on economic and military motives:
		\end{itemize}
		\begin{itemize}
		\item Standard gauge: 1435 mm
		\item Wide gauge
		\begin{itemize}
		\item Russian: 1520 mm
		\item Indian: 1676 mm
		\item Iberian: 1668 mm
		\end{itemize}
		 \item Narrow gauge
		 \begin{itemize}
		\item Cape: 1067 mm
		\item Meter: 1000 mm
		\end{itemize}
		\end{itemize}
     	\end{itemize}
     \end{column}
     	\begin{column}[T]{7cm} 
         	\begin{definition}[Track gauge]
            		The track gauge is the distance between the rails, measured $(14{,}5 \pm 0{,}5) \, \mathrm{mm}$ below top of rail. 
        		\end{definition}
		\begin{definition}[Gauge tolerance]
						Depending on network and track, the gauge has to be within certain tolerances. Common in Europe is $\left(1435^{+35}_{-5} \right)\, \mathrm{mm}$. 
        		\end{definition}
     \end{column}
 \end{columns}
 \begin{center}
 \includegraphics[width = 0.5\textwidth]{Trackgauge}
 \end{center}
 \note{
 \begin{itemize}
		\item Wie h\"angen r\"omische Soldaten und der Durchmesser der Space Shuttle-Tanks zusammen?
		\item Bezeichnung indische Loks.
		\end{itemize}
 }
}

\frame{\frametitle{Geographic distribution of track gauges}
\framesubtitle{}
\begin{center}
\includegraphics[width = 0.9\textwidth]{Railgaugeworld}\rotatebox{90}{{\tiny \color{gray} Source: Wikimedia Commons/Jackdude101}}
\end{center}
}

\subsection{Introduction to running stability}

\frame{\frametitle{Hunting oscillation}
\begin{center}
\begin{tikzpicture}[scale = 0.9, domain=0:4.8]
\path[ultra thick, draw = blue!80!black, name path = track1] (0,0) -- (12,0);
\draw[ultra thick, draw = blue!80!black, name path = track2] (0,4) -- (12,4);
\only<1>{
\begin{scope}[shift = {(0,-.2)}]
\draw[thick, fill = gray!30, fill opacity = .5] (.5, -.3) -- +(-.2, .6) -- +(2,0.6) -- +(1.8,0) -- +(0,0);
\draw[thick, fill = gray!30, fill opacity = .5] (.5, 4.3) -- +(-.2, -.6) -- +(2,-0.6) -- +(1.8,0) -- +(0,0);
\draw[thick, fill = gray!30, fill opacity = .5] (1.2, .3) -- +(0, 3.4) -- +(.4, 3.4) -- +(.4,0) -- +(0,0);
\end{scope}
}
\only<2>{
\begin{scope}[shift = {(0,-.2)}]
\draw[name path = ws1, thick, fill = gray!30, fill opacity = .5] (.5, -.3) -- +(-.2, .6) -- +(2,0.6) -- +(1.8,0) -- +(0,0);
\path [name intersections={of = track1 and ws1}, draw = red!80!black];
\coordinate (A)  at (intersection-1);
\coordinate (B)  at (intersection-2);
\draw[ultra thick, red] (A) -- (B)node[pos =.5, below] {R};
\draw[thick, red] (A) -- +(0,4);
\draw[thick, red] (B) -- +(0,4);
\draw[name path = ws2, thick, fill = gray!30, fill opacity = .5] (.5, 4.3) -- +(-.2, -.6) -- +(2,-0.6) -- +(1.8,0) -- +(0,0);
\path [name intersections={of = track2 and ws2}, draw = red!80!black];
\coordinate (C)  at (intersection-1);
\coordinate (D)  at (intersection-2);
\draw[ultra thick, red] (C) -- (D) node[pos =.5, above] {r};

\draw[thick, fill = gray!30, fill opacity = .5] (1.2, .3) -- +(0, 3.4) -- +(.4, 3.4) -- +(.4,0) -- +(0,0);
\end{scope}
}
\only<3->{
\begin{scope}[shift = {(0,-.2)}]
\draw[thick, fill = gray!30, fill opacity = .5] (.5, -.3) -- +(-.2, .6) -- +(2,0.6) -- +(1.8,0) -- +(0,0);
\draw[thick, fill = gray!30, fill opacity = .5] (.5, 4.3) -- +(-.2, -.6) -- +(2,-0.6) -- +(1.8,0) -- +(0,0);
\draw[thick, fill = gray!30, fill opacity = .5] (1.2, .3) -- +(0, 3.4) -- +(.4, 3.4) -- +(.4,0) -- +(0,0);
\end{scope}
\begin{scope}[shift = {(3,0)}, rotate = 3]
\draw[thick, fill = gray!30, fill opacity = .5] (.5, -.3) -- +(-.2, .6) -- +(2,0.6) -- +(1.8,0) -- +(0,0);
\draw[thick, fill = gray!30, fill opacity = .5] (.5, 4.3) -- +(-.2, -.6) -- +(2,-0.6) -- +(1.8,0) -- +(0,0);
\draw[thick, fill = gray!30, fill opacity = .5] (1.2, .3) -- +(0, 3.4) -- +(.4, 3.4) -- +(.4,0) -- +(0,0);
\end{scope}
}
\only<4->{
\begin{scope}[shift = {(6, .2)}]
\draw[thick, fill = gray!30, fill opacity = .5] (.5, -.3) -- +(-.2, .6) -- +(2,0.6) -- +(1.8,0) -- +(0,0);
\draw[thick, fill = gray!30, fill opacity = .5] (.5, 4.3) -- +(-.2, -.6) -- +(2,-0.6) -- +(1.8,0) -- +(0,0);
\draw[thick, fill = gray!30, fill opacity = .5] (1.2, .3) -- +(0, 3.4) -- +(.4, 3.4) -- +(.4,0) -- +(0,0);
\end{scope}
}
\only<5->{
\begin{scope}[shift = {(9,0)}, rotate = -3]
\draw[thick, fill = gray!30, fill opacity = .5] (.5, -.3) -- +(-.2, .6) -- +(2,0.6) -- +(1.8,0) -- +(0,0);
\draw[thick, fill = gray!30, fill opacity = .5] (.5, 4.3) -- +(-.2, -.6) -- +(2,-0.6) -- +(1.8,0) -- +(0,0);
\draw[thick, fill = gray!30, fill opacity = .5] (1.2, .3) -- +(0, 3.4) -- +(.4, 3.4) -- +(.4,0) -- +(0,0);
\end{scope}
}
\only<6>{
\draw[thick, dashed, color=green!70!black]   plot (1.4+6/3.14*\x,{2-cos(\x r)});
}
\end{tikzpicture}
\end{center}
}

