% !TEX root = SFV-14033_SFT1.tex
\section{Personenfahrzeuge}
\frame{\sectionpage}

\frame{\frametitle{Einf\"uhrung}
\framesubtitle{Die Eisenbahn verkauft \emph{quality time}!}
\begin{itemize}
\item Anspruchsvolle Fahrg\"aste
\begin{itemize}
	\item Verschiedene Anspr\"uch je nach Verkehrsart
	\item Art und Ausstattung an Gattungsbezeichnung zu erkennen
\end{itemize}
\item Umsetzung als Wagen oder Triebzug
\item Wichtige Aspekte:
\begin{itemize}
	\item Inneneinrichtung und Grundriss
	\item Zugang
	\item Ausstattung
	\item Energieversorgung
	\item Fahrkomfort
	\item Fahrgaststr\"ome
	\item Reisegeschwindigkeit
\end{itemize}
\item In verschiedenen Kulturen verschieden Akzeptanz des Bahnverkehrs!
\end{itemize}
}

\frame{\frametitle{Gattungssystematik}
\framesubtitle{}
\begin{center}
%\tiny
\begin{tabular}{|c|c|c|c|}
\hline
\multicolumn{2}{|c|}{Gattungsbuchstabe} & \multicolumn{2}{|c|}{Kennbuchstabe} \\ \hline
A & Sitzwagen 1. Klasse & m & Reisezugwagen oder \\
& & & Wagen eines Triebzugs \\ \hline
AB & Sitzwagen 1. und 2. Klasse & n/y & Nahverkehrswagen \\ \hline
B & Sitzwagen 2. Klasse & x & S-Bahn-Wendezugwagen \\ \hline
D & Gep\"ackwagen & f & mit F\"uhrerraum \\ \hline
D... & Doppelstockwagen & p & klimatisiert, Gro{\ss}raum \\ \hline
...R & mit K\"uche, Bistro & o & vergr\"o{\ss}erte Abteile \\ \hline
...D & mit Gep\"ackabteil & b & Rollstuhleinrichtungen \\ \hline
WL & Schlafwagen & d & mit Mehrzweckraum \\ \hline
WR & Speisewagen & r & mit Rapidbremse \\ \hline
 &  & h/z & Energieversorgung \\ \hline
\end{tabular}
\end{center}
}

%\subsection{Inneneinrichtung}

\frame{\frametitle{Schutzziel gem. TSI}
\framesubtitle{}
\begin{itemize}
\item „Die für die Betätigung durch die Fahrgäste vorgesehenen Einrichtungen müssen so konzipiert sein, dass sie deren Sicherheit nicht gefährden, wenn sie in einer voraussehbaren Weise betätigt werden, die den angebrachten Hinweisen nicht entspricht.“
\end{itemize}
}


\frame{\frametitle{Inneneinrichtung}
\framesubtitle{}
\begin{itemize}
\item Unterschiedliche Bed\"urfnisse in den verschiedenen Verkehrsarten
\item H\"aufig sehr detailliert Inhalt von Verkehrsausschreibungen
\begin{itemize}
	\item Transportm\"oglichkeiten (Fahrrad, Kinderwagen, Rollst\"uhle,...)
	\item Sitzpl\"atze, Tische
	\item \"Uberwachungssysteme (CCTV)
\end{itemize}
\item Einstieg
\begin{itemize}
	\item Fernverkehr: Wagenende
	\item Regional-, Nahverkehr: Dritteleinstieg (oder h\"aufiger)
\end{itemize}
\item Sitzanordnungen
\begin{itemize}
	\item Abteil: 4, 5, 6 Sitze je Abteil, Seitengang
	\item Gro{\ss}raum: i.d.R. 3 oder 4 Sitzpl\"atze je Reihe, Mittelgang
	\item In UK, China: 5 Sitzpl\"atze je Reihe
\end{itemize}
\item Sitzplatzanzahl: Effizienz dominiert
\end{itemize}
}

%\subsection{Barrierefreiheit}
\frame[allowframebreaks]{\frametitle{Barrierefreiheit}
\framesubtitle{Transversale PRM TSI \textit{people with reduced mobility} stellt Anforderungen dar.}
\begin{itemize}
\item Definition People with reduced mobility
\begin{itemize}
	\item Personen, die mit der Nutzung von Eisenbahnen (Fahrzeuge und Infrastruktur) Schwierigkeiten haben
\end{itemize}
\item Au{\ss}ent\"ur:
\begin{itemize}
	\item Kontrast zum Fahrzeug, Bedienung auf oder neben dem T\"urblatt, Sicht- und H\"orbare Warnung bei Bet\"atigung
	\item Lichte Weite mindestens 800 mm (HST), mindestens 1000 mm (CR)
\end{itemize}
\item Zustiegshilfe
\begin{itemize}
	\item W\"unschenswert: angepasste Fahrzeuge f\"ur Infratruktur
	\item Sonst: Rampen, \"Uberfahrbr\"ucken etc.
\end{itemize}
\item Inneneinrichtung
\begin{itemize}
	\item Verf\"ugbarkeit von Haltegriffen, Vorrangsitzen (10\%)
	\item Rollstuhlp\"atze: 1 ($L_{Zug} < 30$ m) bis 4 ($L_{Zug} > 300$ m)
	\item Hilferufvorrichtung
	\item Rampen eingeschr\"ankt zul\"assig
	\item Haltestangen, D = (30...40) mm
\end{itemize}
\item Toiletten
\begin{itemize}
	\item Vorhandensein einer Universaltoilette
\end{itemize}
\item Fahrgastinformation:
\begin{itemize}
	\item Piktogramme (max. 5 zusammen)
	\item Taktile Informationen
	\item Displays etc. von 51\% der Fahrgastpl\"atze und allen Rollstuhlpl\"atzen lesbar
\end{itemize}
\end{itemize}
}

%\subsection{Energieversorgung}
\frame{\frametitle{Energieversorgung}
\framesubtitle{}
\begin{itemize}
\item In Wagen:
\begin{itemize}
	\item Dominierend: Zugsammelschiene gem\"a{\ss} UIC 552
	\item Verschiedene Spannungen und Frequenzen, z.B.:
	\begin{itemize}
		\item AC 1000 V 16,7 Hz
		\item AC 1500 V 50 Hz
		\item DC 1500 V
		\item DC 3000 V
	\end{itemize}
	\item Strom: (800...1000) A (je Kupplung 600 A)
	\item Stromart erfordert Gl\"attung/Wechselrichtung
	\item Vereinzelt Achgeneratoren
\end{itemize}
\item In Triebz\"ugen:
\begin{itemize}
	\item Verbindung im Rahmen der Fahrzeugverdrahtung
	\item Bordnetzspannung 24 V, 72 V, 110 V (je -30\%/+25\%)
\end{itemize}
\end{itemize}
}

\frame{\frametitle{T\"uren und T\"ursteuerung}
%\framesubtitle{``Die beiden gef\"ahrlichen Schnittstellen mit der Bev\"olkerung sind Bahn\"uberg\"ange und T\"uren''}
\begin{columns}[t] 
     \begin{column}[T]{6cm} 
     	\begin{itemize}
     		\item Wichtige Aspekte:
		\begin{itemize}
		\item \"Offnungsweite
		\item Druckert\"uchtigung
		\item Festigkeit (insb. HST)
		\item Sicherheit
		\end{itemize}
		\item Bauarten:
		\begin{itemize}
		\item Drehfaltt\"ur
		\item Schwenkschiebet\"ur in verschiedenen Bauarten
		\end{itemize}
		\item T\"ursteuerung:
		\begin{itemize}
		\item Verschiedene Verfahren (Automatisierung):
		\begin{itemize}
		\item T\"ursicherung
		\item TB 0%: %T\"urblockierung ab 0 km/h
		\item SAT%: Selbstabfertigung durch Tf
		\item TAV%: Technikbasiertes Abfertigungsverfahren 
		\end{itemize}
		\end{itemize}
     	\end{itemize}
     \end{column}
     	\begin{column}[T]{6cm} 
         	\begin{center}
            		\includegraphics[width=0.4\textwidth]{Drehfalttuer} \rotatebox{90}{\color{gray} \tiny Quelle: Wikimedia/LosHawlos}\\
		\includegraphics[width=0.8\textwidth]{SST} \rotatebox{90}{\color{gray} \tiny Quelle: Wikimedia/Lief J\"orgensen}
        		\end{center}
     \end{column}
 \end{columns}
}

\frame{\frametitle{Klimatisierung}
\framesubtitle{Die Aufgaben Heizen, Bel\"uften und Klimatisieren werden h\"aufig integriert (HVAC).}
\begin{itemize}
\item Ausf\"uhrungen:
\begin{itemize}
	\item Heute dominierend: elektrische Energieversorgung
	\item Noch im Bestand: Dampf/elektrische Heizungen, \"Olheizungen
	\item F\"ur K\"uhlung: K\"uhlmittel- und Kaltluftanlagen 
\end{itemize}
\item Aufgaben:
\begin{itemize}
	\item Heizen: Innenraumtemperatur auf bestimmtem Niveau halten
	\item Bel\"uftung: ben\"otigte Luftmenge zuf\"uhren
	\item Klimatisieren: Innenraumtemperatur auf bestimmtem Niveau halten
\end{itemize}
\item Herausforderungen:
\begin{itemize}
	\item Gro{\ss}e Fahrzeugfl\"achen und -scheiben
	\item Hohe, schwankende Personenzahlen
	\item Installationsraum
	\item T\"ur\"offnung
	\item Feuchtigkeitszufuhr (nasse Reisende)
	\item Zugfreiheit
\end{itemize}
\end{itemize}
}

\frame{\frametitle{Fahrgastnotruf}
\framesubtitle{Der Fahrgastnotruf l\"ost die Notbremse bei TSI-konformen Fahrzeugen ab.}
\begin{columns}[t] 
     \begin{column}[T]{6cm} 
     	\begin{itemize}
     		\item Ausstattung:
		\begin{itemize}
		\item Jedes Abteil, Vorr\"aume und alle anderen abgetrennten Bereiche ausser Toiletten und \"Uberg\"ange
		\item Sichtbar und gekennzeichnet
		\end{itemize}
		\item Alarm kann nicht abgebrochen werden
		\item Alarm wird Tf visuell und akustisch angezeigt
		\item Tf kann best\"atigen, dies wird Fahrg\"asten mitgeteilt
		\item Kommunikation mit Tf
		\item R\"ucksetzung durch Zugpersonal
     	\end{itemize}
     \end{column}
     	\begin{column}[T]{6cm} 
         	\begin{center}
            		\includegraphics[width=0.65\textwidth]{Notbremse}
        		\end{center}
     \end{column}
 \end{columns}
}

\frame{\frametitle{Fahrgastinformationssysteme}
\framesubtitle{}
\begin{itemize}
\item Aufgaben:
\begin{itemize}
	\item Information des Reisenden: Zuglauf, n\"achster HAlt, etc.
	\item Kommunikation (betrieblich und \"offentlich, Mobilfunk-Repeater, WLAN, ...)
	\item Unterhaltung
	\item Kommunikation im Notfall (siehe SFT2: Notbremsanforderung)
\end{itemize}
\item Umsetzung:
\begin{itemize}
	\item Anzeigen
	\item Elektroakustische Anlage (ELA)
\end{itemize}
\end{itemize}
}

\frame{\frametitle{Fahrgastinformationssysteme}
\framesubtitle{}
\begin{columns}[t] 
     \begin{column}[T]{6cm} 
     	\begin{itemize}
     		\item Aufgaben:
\begin{itemize}
	\item Information des Reisenden: Zuglauf, n\"achster HAlt, etc.
	\item Kommunikation (betrieblich und \"offentlich, Mobilfunk-Repeater, WLAN, ...)
	\item Unterhaltung
	\item Kommunikation im Notfall (siehe SFT2: Notbremsanforderung)
\end{itemize}
\item Umsetzung:
\begin{itemize}
	\item Anzeigen
	\item Elektroakustische Anlage (ELA)
\end{itemize}

     	\end{itemize}
     \end{column}
     	\begin{column}[T]{6cm} 
         	\begin{center}
            		\includegraphics[width=0.8\textwidth]{FIS}
        		\end{center}
     \end{column}
 \end{columns}
}

